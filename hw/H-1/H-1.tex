\documentclass{article}
\usepackage{../fasy-hw}
\usepackage{ wasysym }
\usepackage{proof}

%% UPDATE these variables:
\renewcommand{\hwnum}{1}
\title{Advanced Algorithms, Homework 1}
\author{Kevin Browder}
\collab{n/a}
\date{due: 27 August 2020}

\begin{document}

\maketitle

This homework assignment is due on 27 August 2020, and should be
submitted as a single PDF file to D2L and to Gradescope.

General homework expectations:
\begin{itemize}
    \item Homework should be typeset using LaTex.
    \item Answers should be in complete sentences and proofread.
\end{itemize}

\nextprob
\collab{n/a}

Answer the following questions:
\begin{enumerate}
    \item What is your elevator pitch? 
    \item What was your favorite CS class so far, and why? 
    \item What was your least favorite CS class so far, and why? 
    \item Why are you interested in taking this course? 
    \item What is your biggest academic or research goal for this semester (can
        be related to this course or not)? 
    \item What do you want to do after you graduate? 
    \item What was the most challenging aspect of your coursework last semester
        after the university transitioned to online? 
    \item What went well last semester for you after the university transitioned
        to online? 
\end{enumerate}

\paragraph{Answer}

% ============================================

\begin{enumerate}
	\item I am 22 years old and grew up in the Bitteroot and Hailey, Idaho. I am a computer science major and like to run, climb and ski. 
	\item My favorite CS class so far was ESOF 322 with Clem.
	\item My least favorite CS class so far was Programming Languages.
	\item I am looking forward to integrating a TensorFlow model into a mobile app for YERC. 
	\item Play in the mountains and hopefully move to New Zealand/Australia.
	\item My motivation was at an all time low. 
	\item I enjoyed spending time with my family.
\end{enumerate}

% ============================================

\nextprob
\collab{n/a}

Please do the following:
\begin{enumerate}
    \item Write this homework in LaTex.
        Note: if you have not used LaTex before and this is an
        issue for you, please contact the instructor or TA.
    \item Update your photo on D2L to be a recognizable headshot of you.
    \item Sign up for the class discussion board.
\end{enumerate}

\paragraph{Answer}

% ============================================

I did the tasks.

% ============================================


\nextprob
\collab{}

    In this class,
    please properly cite all resources that you use.
    To refresh your memory on what plagiarism is,
    please
    complete the plagiarism tutorial found here:
    \url{http://www.lib.usm.edu/plagiarism_tutorial}.
    If you have observed plagiarism or cheating in a classroom (either as an
    instructor or as a student), explain the situation and how it made you
    feel.  If you have not experienced plagiarism or cheating or if you would
    prefer not to reflect on a personal experience, find a news
    article about plagiarism or cheating and explain how you would feel if you
    were one of the people involved.


\paragraph{Answer}

% ============================================

I have TAed for first year classes for 3 semesters now and had some instances of cheating. I had one instance where a student copy and pasted out of Stack Overflow right in front of me. I get looking at other code to get inspiration or ideas but copy and pasting right in front of the teacher is a bit over the top. It was very annoying. 

% ============================================



\nextprob
Prove the following statement: Every tree with one or more nodes/vertices has
exactly $n-1$ edges.

\paragraph{Answer}

% ============================================

\begin{proof}
	By Induction. Let n be the number of vertices in a tree (T). \\
	Now we will use Induction to prove this is true for all $n$.\\
	Base Case: n=1 will have 0 edges this is true. \\
	Now we will assume this holds true for $T$ with $n$ edges. \\ 
	We will now prove this holds true for $n=m+1$. \\
	The number of edges is m-1 + number of edges required to add 1 node by the inductive assumption. 
	The number of edges to add a node is 1 by the definition of a tree. Each node must be connected and the graph must be acyclic. This means that you must add one edge when a node is added.
	Therefore by induction a tree with one or more nodes has exactly $n-1$ edges. 
	
\end{proof}

% ============================================



\nextprob
Use the definition of big-O notation to prove that $f(x)=n^2 + 3n +2$ is
$O(n^2)$.

\paragraph{Answer}

% ============================================

\begin{proof}
	By the definition of Big-), $f(x)$ is $O(n^2)$ if $f(x) \leq c*n^2$ fro some $n \geq n_0$. Let us check this condition: if $n^2 + 3n + 2 \leq c*n^3$ then $1 + \cfrac{3}{n} + \cfrac{2}{n^2} \leq c$. Therefore, the Big-O condition holds for $n\geq n_0 = 1$ and $c \geq 6 = (1 + 3 + 2)$. Larger values of $n_0$ will result in smaller factors c but the statement has been proved. 
	
\end{proof}

% ============================================



\nextprob
Consider the \textsc{RightAngle} algorithm on page 8 of the textbook.
\begin{enumerate}
    \item When we design an algorithm, we design the algorithm to solve a
        problem or answer a question.  What is the problem that this algorithm
        solves?
    \item Prove that the algorithm terminates.
\end{enumerate}

\paragraph{Answer}

% ============================================

\begin{enumerate}
	\item Given a line $l$ and point $P$, the algorithm constructs a new line $l_n$ that is perpendicular to $l$ and passed through point $P$. 
	\item The \textsc{RightAngle} algorithm does not contain a loop or recursion. It is a simple linear program so as long as the CIRCLE and INTERSECT methods terminate then the algorithm will terminate. It must terminate because it cannot do anything else. 
\end{enumerate}

% ============================================



\nextprob
Consider the following statement: If $a$ and $b$ are both even numbers, then $ab$ is
an even number.
\begin{enumerate}
    \item What is the definition of an odd number? 
    \item What is the definition of an even number?  
    \item What is the contrapositive of this statement? 
    \item What is the converse of this statement? 
    \item Prove this statement.
\end{enumerate}

\paragraph{Answer}

% ============================================

\begin{enumerate}
	\item Odd Number: n is odd if $n = 2k+1$ where $k$ is an integer.
	\item Even Number: n is even if $n = 2k$ where $k$ is an integer.
	\item If $ab$ is an odd number, than $a$ and $b$ are both odd.
	\item If $ab$ is an even number, then $a$ and $b$ are both even number. 
	\item 
	\begin{proof}Since $a$ and $b$ are even they can be rewritten as $a = 2m$ and $b = 2n$.\\ So, $ab = 2m(2n) = 2(2mn)$ which is even by the definition of an even number.
	\end{proof} 
\end{enumerate}

% ============================================



\end{document}

